%% LyX 2.3.6.1 created this file.  For more info, see http://www.lyx.org/.
%% Do not edit unless you really know what you are doing.
\documentclass[11pt,english]{article}
\usepackage[T1]{fontenc}
\usepackage[latin9]{inputenc}
\usepackage[a4paper]{geometry}
\geometry{verbose,lmargin=2.5cm,rmargin=2.5cm}
\usepackage{amsmath}
\usepackage{amssymb}

\makeatletter

%%%%%%%%%%%%%%%%%%%%%%%%%%%%%% LyX specific LaTeX commands.
%% Because html converters don't know tabularnewline
\providecommand{\tabularnewline}{\\}

%%%%%%%%%%%%%%%%%%%%%%%%%%%%%% User specified LaTeX commands.
\date{}

\makeatother

\usepackage{babel}
\begin{document}
\title{Conversion of OpenCV to OpenTrackIO (OpenLensIO)\\
lens calibration parameters}

\maketitle
If OpenCV lens calibration parameters are available for a camera/lens
combination, how are these parameters converted to corresponding OpenTrackIO
(OpenLensIO) lens calibration parameters?

OpenCV definitions of parameters can be found here:
\begin{quote}
\texttt{https://docs.opencv.org/4.x/d9/d0c/group\_\_calib3d.html}
\end{quote}
whereas OpenTrackIO/OpenLensIO definitions of parameters can be found
here:
\begin{quote}
\texttt{https://www.opentrackio.org/}
\end{quote}
(search for ``OpenLensIO mathematical lens model'') and will be
referred to from now on as just OpenTrackIO parameters (omitting OpenLensIO).\\

A few additional parameters for OpenTrackIO will be helpful and are
defined first:

The radius $r_{u}$ (distance from origin) of undistorted screen coordinates
$\epsilon_{u}$:

\[
r_{u}=\sqrt{{\epsilon'_{x,u}}^{2}+{\epsilon'_{y,u}}^{2}}
\]

The pixel coordinates $\epsilon_{s}$, here expressed as shader coordinates
$\varepsilon_{shader}$ but relative to the image center (not the
upper left corner) and using a texture that has the same resolution
as the camera image with width $w_{shader}$ and height $h_{shader}$
(so the camera image is $w_{shader}$ pixels wide and $h_{shader}$
pixels high, and $h_{shader}$ is different from $w_{shader}$ as
opposed to the assumption of a square texture in the OpenLensIO documentation):

\[
\epsilon_{x,s}=\varepsilon_{x,shader}-\frac{w_{shader}}{2}=w_{shader}\cdot\frac{\epsilon_{x,d}}{w}=\frac{w_{shader}}{w}(\epsilon'_{x,d}+\Delta\mathbb{P}_{x})
\]

\[
\epsilon_{y,s}=\varepsilon_{y,shader}-\frac{h_{shader}}{2}=h_{shader}\cdot\frac{\epsilon_{y,d}}{h}=\frac{h_{shader}}{h}(\epsilon'_{y,d}+\Delta\mathbb{P}_{y})
\]

Parameters $l_{1},...,l_{6}$ are the radial, $q_{1},q_{2}$ the tangential
distortion parameters of the distortion function, which is the inverse
of undistortion function U (formerly named D) defined in the OpenLensIO
documentation with $k_{1},...,k_{6}$ as its radial, $p_{1},p_{2}$
as its tangential distortion parameters (not to be confused with the
OpenCV parameters of the same name but referring to the inverse operation).\\

Using these additional definitions, the following table lists corresponding
parameters for both lens models:
\begin{center}
\begin{tabular}{|c|c|c|}
\hline 
OpenCV & Parameter & OpenTrackIO\tabularnewline
\hline 
\hline 
$x',y'$ & undistorted screen coordinates & $\epsilon'_{x,u},\epsilon'_{y,u}$\tabularnewline
\hline 
$r$ & radius of undistorted coordinates & $r_{u}$\tabularnewline
\hline 
$x'',y''$ & distorted screen coordinates & $\epsilon'_{x,d},\epsilon'_{y,d}$\tabularnewline
\hline 
$u,v$ & pixel coordinates & $\epsilon_{x,s},\epsilon_{y,s}$\tabularnewline
\hline 
$f_{x},f_{y}$ & focal length & $F$\tabularnewline
\hline 
$c_{x},c_{y}$ & principal point / projection offset & $\Delta\mathbb{P}_{x},\Delta\mathbb{P}_{x}$\tabularnewline
\hline 
$k_{1},...,k_{6}$ & radial distortion & $l_{1},...,l_{6}$\tabularnewline
\hline 
$p_{1},p_{2}$ & tangential distortion & $q_{1},q_{2}$\tabularnewline
\hline 
\end{tabular}\medskip{}
\par\end{center}

The following equations follow straight from the parameter definitions:
\begin{center}
\begin{tabular}{ccccc}
$\epsilon'_{x,u}=F\cdot x'$ &  & $\epsilon'_{x,d}=F\cdot x''$ &  & $\epsilon_{x,s}=u$\tabularnewline
\noalign{\vskip\doublerulesep}
$\epsilon'_{y,u}=F\cdot y'$ &  & $\epsilon'_{y,u}=F\cdot y''$ &  & $\epsilon_{y,s}=v$\tabularnewline
\noalign{\vskip\doublerulesep}
$r_{u}=F\cdot r$ &  &  &  & \tabularnewline
\noalign{\vskip\doublerulesep}
\end{tabular}
\par\end{center}

Focal length and projection offset parameters can be converted as
follows:
\[
f_{x}\cdot x''+c_{x}=u=\epsilon_{x,s}=\frac{w_{shader}}{w}(\epsilon'_{x,d}+\Delta\mathbb{P}_{x})=\frac{w_{shader}}{w}\cdot F\cdot x''+\frac{w_{shader}}{w}\cdot\Delta\mathbb{P}_{x}
\]

Since this equation is true for any value of $x''$ it follows that
\[
F=\frac{w}{w_{shader}}\cdot f_{x}\qquad\wedge\qquad\Delta\mathbb{P}_{x}=\frac{w}{w_{shader}}\cdot c_{x}
\]

Likewise in y direction:
\[
f_{y}\cdot y''+c_{y}=v=\epsilon_{y,s}=\frac{h_{shader}}{h}(\epsilon'_{y,d}+\Delta\mathbb{P}_{y})=\frac{h_{shader}}{h}\cdot F\cdot y''+\frac{h_{shader}}{h}\cdot\Delta\mathbb{P}_{y}
\]
\[
\Rightarrow\qquad F=\frac{h}{h_{shader}}\cdot f_{y}\qquad\wedge\qquad\Delta\mathbb{P}_{y}=\frac{h}{h_{shader}}\cdot c_{y}
\]

Comparing individual components in the lens distortion formulas leads
to the following conversions for distortion parameters, starting with
radial distortion parameters in the numerator of the formula:
\[
k_{1}\cdot r^{2}=l_{1}\cdot r_{u}^{2}=l_{1}\cdot F^{2}\cdot r^{2}\qquad\Rightarrow\qquad l_{1}=\frac{k_{1}}{F^{2}}
\]
\[
k_{2}\cdot r^{4}=l_{3}\cdot r_{u}^{4}=l_{3}\cdot F^{4}\cdot r^{4}\qquad\Rightarrow\qquad l_{3}=\frac{k_{2}}{F^{4}}
\]
\[
k_{3}\cdot r^{6}=l_{5}\cdot r_{u}^{6}=l_{5}\cdot F^{6}\cdot r^{6}\qquad\Rightarrow\qquad l_{5}=\frac{k_{3}}{F^{6}}
\]

Likewise for the radial distortion parameters in the denominator of
the formula:
\[
l_{2}=\frac{k_{4}}{F^{2}}\qquad\qquad l_{4}=\frac{k_{5}}{F^{4}}\qquad\qquad l_{6}=\frac{k_{6}}{F^{6}}
\]

Finally, the same type of comparison leads to this conversion of tangential
distortion parameters:
\[
p_{1}\cdot x'\cdot y'=q_{1}\cdot\epsilon'_{x,u}\cdot\epsilon'_{y,u}=q_{1}\cdot F\cdot x'\cdot F\cdot y'\qquad\Rightarrow\qquad q_{1}=\frac{p_{1}}{F^{2}}
\]
\begin{align*}
p_{2}(r^{2}+2{x'}^{2}) & =q_{2}(r^{2}+2{\epsilon'_{x,u}}^{2})\\
 & =q_{2}((F\cdot r)^{2}+2(F\cdot x')^{2})\\
 & =q_{2}\cdot F^{2}(r^{2}+2{x'}^{2})\qquad\Rightarrow\qquad q_{2}=\frac{p_{2}}{F^{2}}
\end{align*}
\medskip{}

This is a summary of all conversions:\\

\begin{tabular}{lccc}
\noalign{\vskip\doublerulesep}
focal length: & \multicolumn{3}{c}{$F=\frac{w}{w_{shader}}\cdot f_{x}=\frac{h}{h_{shader}}\cdot f_{y}$}\tabularnewline[\doublerulesep]
\noalign{\vskip\doublerulesep}
\noalign{\vskip\doublerulesep}
projection offset: & $\Delta\mathbb{P}_{x}=\frac{w}{w_{shader}}\cdot c_{x}\quad$ &  & $\Delta\mathbb{P}_{y}=\frac{h}{h_{shader}}\cdot c_{y}\quad$\tabularnewline[\doublerulesep]
\noalign{\vskip\doublerulesep}
\noalign{\vskip\doublerulesep}
radial distortion: & $l_{1}=\frac{k_{1}}{F^{2}}$ & $l_{3}=\frac{k_{2}}{F^{4}}$ & $l_{5}=\frac{k_{3}}{F^{6}}$\tabularnewline[\doublerulesep]
\noalign{\vskip\doublerulesep}
\noalign{\vskip\doublerulesep}
 & $l_{2}=\frac{k_{4}}{F^{2}}$ & $l_{4}=\frac{k_{5}}{F^{4}}$ & $l_{6}=\frac{k_{6}}{F^{6}}$\tabularnewline[\doublerulesep]
\noalign{\vskip\doublerulesep}
\noalign{\vskip\doublerulesep}
tangential distortion: & $q_{1}=\frac{p_{1}}{F^{2}}$ & $q_{2}=\frac{p_{2}}{F^{2}}$ & \tabularnewline[\doublerulesep]
\noalign{\vskip\doublerulesep}
\end{tabular}\\
\medskip{}

\emph{Important note:} The OpenCV parameters implicitly depend on
the pixel resolution of the camera image $(w_{shader},h_{shader})$
and the size of the image sensor $(w,h)$. By applying the conversions
above, the resulting OpenTrackIO parameters do \emph{not} depend on
camera image resolution and image sensor size anymore. It is therefore
possible (within practical limits) to replace the camera or switch
it to a different sensor resolution (different sampling mode) without
the need to recalibrate all lens parameters!
\end{document}
